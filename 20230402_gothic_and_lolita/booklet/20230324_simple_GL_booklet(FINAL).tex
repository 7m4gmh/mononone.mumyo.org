\documentclass[a6paper, 9pt, openright, titlepage, twoside]{ltjsarticle}
%%\documentclass[twoside, a6paper, 9pt, openright]{jlreq}

\usepackage[top=13truemm,bottom=4truemm,margin=15truemm]{geometry}
%\setlength{\headheight}{10truemm}
%\setlength{\topmargin}{-25truemm}%% \topmarignを-0.5インチにする
%\setlength{\headsep}{0pt}
%\setlength{\marginparsep}{0pt} %本文領域と傍注領域との空き
%\setlength{\footskip}{0truemm}
\usepackage[hiragino-pro]{luatexja-preset}
\usepackage{graphicx}
\usepackage{multirow}
%%\usepackage{amssymb,amsmath}
\usepackage{url}
\usepackage[x-1a1]{pdfx}
\usepackage[pdfbox]{gentombow}
\usepackage{caption}



\title{ Gothic and Lolita } 
\author{mumyo} 
\date{2023年4月2日}

\begin{document}
%%%%%%%%%%\maketitle

\newpage
\thispagestyle{empty}
 %% 全角空白

\begin{center}
\vspace*{\stretch{2}}
{\HUGE  \hspace{2\zw}  \ttfamily Gothic \\ and  \vspace{1\zw} \\ Lolita}
%{\Large  \hspace{2\zw}  \ttfamily Gothic \\ and  \vspace{1\zw} \\ Lolita}


\vspace{\stretch{2}}
 {\Huge \hspace{2\zw}  \ttfamily mumyo }
%%{\Large \hspace{2\zw}  \ttfamily mumyo }

\vspace{\stretch{2}}

\end{center}


\newpage



\thispagestyle{empty}
%%%\setcounter{page}{1}

%%\subsection*{もくじ}

\vspace*{-16mm} %%先頭行の位置変更
\subsubsection*{プログラム}
演奏: 成田達輝(ヴァイオリン)%%%%\footnote{演目の順序は変更することがあります。 }
%\vspace{-3mm}

\begin{enumerate}
\item J.S.バッハ / 無伴奏ヴァイオリンのためのパルティータ 第3番ホ長調 BWV1006より ロンド形式のガヴォット {\ttfamily [p.\pageref{bwv1006}]} 
\item 梅本 / コピー・アンド・ペースト, 大量生産/消費された 不規則/不完全な形状のプラスチック真珠そして私。(2022)  {\ttfamily [p.\pageref{copy}]} 
\item 梅本 / インベリッシュ・ミー! (2022)  {\ttfamily [p.\pageref{embellish}]} 
\item 山根 / リボン集積  (2022)  {\ttfamily [p.\pageref{ribbon}]} 
\item 山根 / パニエ、美学  (2022)   {\ttfamily [p.\pageref{pannier}]}
\end{enumerate}
\vspace{-3mm}
 \hspace{8\zw}    * * *%%{\gt 休憩} 
 \vspace{-5mm}
\begin{enumerate}
\setcounter{enumi}{5}
\item 梅本 / メルト・ミー! (2022) {\ttfamily [p.\pageref{melt}]} 
\item J.S.バッハ / 無伴奏ヴァイオリンソナタ 第2番イ短調 BWV1003より アレグロ {\ttfamily [p.\pageref{bwv1003}]} 
\item 山根 / リボンの血肉と蒸気  (2022) {\ttfamily [p.\pageref{ribbonflesh}]}  
\item 梅本 / 廃墟・秋葉原のアリス 1, 2  (2022)  {\ttfamily [p.\pageref{alice}]} 
\item 山根 / 黒いリボンをつけたブーレ (2022)  {\ttfamily [p.\pageref{bouree}]} 
\end{enumerate}
\vspace{-3mm}
 \hspace{8\zw}    * * *%%{\gt 休憩} 
 \vspace{-5mm}
\begin{enumerate}
\setcounter{enumi}{10}
\item  トークセッション \footnote{14時公演ゲスト  朝藤りむ(デザイナー), 17時公演ゲスト  NABEchan(イラストレーター)}
\end{enumerate}

\nopagebreak

\vspace{-5mm}
\subsubsection*{解説}
\begin{itemize}
\item 梅本佑利 / 少女、アリス、山根明季子 {\ttfamily [p.\pageref{statement}]}  
\item 鈴木真理子  / 小説「ロリータ」から43年後、「ゴスロリ」は闇の姫服として誕生した  {\ttfamily [p.\pageref{suzuki}]}  
\end{itemize}
\newpage

 
\subsection*{
無伴奏ヴァイオリンのためのパルティータ第3番 ホ長調 BWV1006より ロンド形式のガヴォット
}\label{bwv1006}
{\small 作曲: ヨハン・セバスチャン・バッハ (1685-1750)}

ヴァイオリン学習者にとっては登竜門であり同時に生涯通じて演奏されるであろう「無伴奏ヴァイオリンのためのソナタとパルティータ全6曲」(BWV1001-1006)は、1720年ごろに書かれました。この頃バッハは、無類の音楽好きだったケーテン侯レオポルトの庇護のもと、ケーテンの宮廷楽長として多くの世俗音楽の傑作を残しました。

このヴァイオリン1本で演奏される、しかしとても多声的な楽曲集の中から、今回は、とりわけ明るく華やかとされるパルティータ(組曲の意)の第3番から有名な「ロンド形式のガヴォット」を演奏します。曲のタイトルにある「ガヴォット」は16世紀フランスのブルターニュ地方発祥の民衆のためのダンスで、ブランルという円になって踊る曲と共に当時人気がありました。

バッハは、これをヴァイオリンのみで演奏しても聴き映えがするようにロンド(曲のテーマが何度も廻ること)形式を加えることにより、フレーズの展開に新鮮さを加えたのかなと思います。 (文・成田達輝)


\subsection*{
コピー・アンド・ペースト, 大量生産/消費された 不規則/不完全な形状のプラスチック真珠そして私。 \\
Copy and paste, mass produced/consumed irregular/imperfectly shaped plastic pearls and Me. (2022) 
}\label{copy}
{\small 作曲: 梅本佑利}

コンピュータ上で行われたこれらの作曲における、いわゆる「コピペ」の多用を表題とし、無限に生産、消費され続けるチープなプラスチック製の「歪んだ真珠」=「バロック」を描く。
過去の文化が歪んだ形(廃墟)となって目の前に現れる様子は、梅本「インベリッシュ・ミー!」、「メルト・ミー!」に共通する。



\subsection*{
インベリッシュ・ミー!  \\
Embellish Me! (2022) }\label{embellish}
{\small 作曲: 梅本佑利}

「インベリッシュ・ミー!」(私を装飾して!)での装飾音符は、装飾される音の原型を留めないほど過剰に扱われる。また、ここでは、モチーフにかけられる微分音的な音高の変化、伸び縮みを「装飾音」とする。この装飾の概念は、DAW上でオーディオのピッチを変調する行為に近い。アリスは現代の少女として、自撮りのフィルターで顔や体を装飾し、自在に変身する。 

「インベリッシュ・ミー!」、「メルト・ミー!」では、ルイス・キャロルによる児童小説「不思議の国のアリス」のモチーフや世界観が根底にある。ここで描かれる「アリス」は、「ゴスロリ」文化にとって象徴的なキャラクターであり、西洋音楽における「少女」の現在地を、西洋文化と、日本的サブカルチャーの両目の視点で更新する存在である。 

それぞれの題名にある ``~ Me'' の語感は、同名の小説に登場する、 ``Drink Me'' (私を飲んで)や、 ``Eat Me'' (私を食べて)と書かれた小瓶やケーキに由来する。アリスはこれらを飲んだり食べたりすることで、体を自在に変化させ、不思議な世界を知恵で生き延びる。 これらの曲は、そんなアリスのメタモルフォーゼを「ゴスロリ」的な装飾として表す。

\subsection*{
リボン集積 \\
Ribbon Accumulation (2022) }\label{ribbon}
{\small 作曲: 山根明季子}

ヨーロッパ伝統の機能和声を下地とし、その主従と秩序、規律、逸脱を紡いだ音楽。リボンという少女的アイコンを、肌感覚を通して音という目に見えないものに抽象化し延々と描くことで作られている。リボンは西洋の音の伝統と重ねられ、その崇高さの奥底・内側にある暴力や残酷性、グロテスクといった様々なものを反芻していく。


%%\newpage
\subsection*{
パニエ、美学 \\
Pannier clothing, Aesthetic (2022) }\label{pannier}
{\small 作曲: 山根明季子}

西洋音楽対位法の完成形、フーガ。その荘厳なるフーガを成立させている構造を放棄して、ばらばらに分解して繰り返し、永遠性への志向のもと機械装置的に円環させる。タイトルのパニエは、ドレスやスカートを膨らませるためのルイ王朝時代に起源を持つ下着であり、日本ではロリータ・ファッションとして独自の形に用いられている。現代日本のストリートでも着用して歩ける丈の短いドレス・西洋装を、切断した音楽形式フーガに重ねる。 



%%\newpage
\subsection*{
メルト・ミー! \\
Melt Me! (2022) }\label{melt}
{\small 作曲: 梅本佑利}

「メルト・ミー!」(私を溶かして!)では、「インベリッシュ・ミー!」にみられた微分音的「装飾」で、溶けるケーキ=バッハを描く。ケーキが溶けるというイメージは、本プロジェクトのために描かれた NABEchan のビジュアルから着想を得た。




\begin{figure}[h]
\centering
\scalebox{0.08}{\includegraphics{meltme_bw.png}} 
\caption*{{\itshape Melt Me!} スケッチ (梅本)}
\end{figure}

%%\newpage

\subsection*{
無伴奏ヴァイオリンソナタ第2番イ短調 BWV1003より 第4楽章 アレグロ
}\label{bwv1003}
{\small 作曲:ヨハン・セバスチャン・バッハ}

「パルティータ第3番ホ長調」(BWV1006)では、ヴァイオリンのE線、1番細く高いミの音を基準に作曲されていることもあり、全体的に重心が高めで軽やかさが目立ちますが、対照的にこのソナタ第2番では、A線・ラの音を基準に作られているため、より音階の上行形が多く、また、ソナタ第1番に見られなかった ``フォルテとピアノの表記によるコントラスト'' が加わり、より作品の緊張度を高めています。 (文・成田達輝)


\subsection*{
リボンの血肉と蒸気 \\
Ribbon flesh and blood and vapor (2022) }\label{ribbonflesh}
{\small 作曲: 山根明季子}

全体がJ.S. バッハ「無伴奏ヴァイオリンソナタ第2番イ短調アレグロ」の切り貼りと加工によってのみ作られている。近代以降ネット社会に至るまで、資本主義が加速する時代の肉体の記憶をテーマにマリア像を想いコラージュを進めた。



%%\newpage
\subsection*{
廃墟・秋葉原のアリス 1, 2 \\
Alice in Abandoned Akihabara 1, 2 (2022) }\label{alice}
{\small 作曲: 梅本佑利}

この作品は、近未来、廃墟となった秋葉原のメイドカフェの世界線である。破壊と再生の街、秋葉原。今から150年前、大火を受けた火除地に「秋葉原」は生まれた。その後、第二次大戦で再び焼け野原となり、闇市から電気街が生まれ、高度経済成長とともに発展。しかし、バブル崩壊で家電市場は奪われ、オタクの街となった。同時期にゴスの文脈から派生したメイド服のロリータは、秋葉原の住人「オタク」に吸収された。そしてオタクの街は外国人に発見され、観光地となった。

今、秋葉原は未曾有のパンデミックによって廃墟化が進む。その先の近未来、あるいは核戦争、経済破綻、巨大地震、隕石の衝突、宇宙人の襲来…で、東京・秋葉原が風化、破壊されたディストピアな世界線。 これは「オタク」と「ゴス」を横断するロリータ、アリスの物語。吸収するオタクとゴスの逆転劇。未来を見つめるSF的な「オタク」視点と、過去を見つめる「ゴス」の廃墟への眼差し、西洋と日本文化、複雑化したシミュレーションのねじれがここにある。


%%\newpage
%%\section*{}
\subsection*{黒いリボンをつけたブーレ \\
Bourrée with black ribbons (2022) }\label{bouree}
{\small 作曲: 山根明季子}

本企画新作の中で最初に作られた音楽。 J.S. バッハ「無伴奏ヴァイオリンのためのパルティータ第1番ロ短調」(BWV1002)より{\itshape Tempo di Bourrée} を下地に、 感覚的にゴシックとロリータについての考察を始めた。 「黒いリボンをつけたブーレ」はヴァイオリン独奏曲として、クラシック音楽におけるルールを素材として扱い、ポップに切り分け、21世紀日本のストリートと ``崇高'' なる西洋の「影」を重ねて繰り返し、黒いリボンを飾りつけるように書かれた。


\newpage


\subsection*{ステートメント「少女、アリス、山根明季子」}\label{statement}
{\small 文: 梅本佑利}\footnote{うめもと・ゆうり, 作曲家}  

山根明季子にとっての最初の委嘱作品である2008年作曲『ケミカルロリイタ』(チューバとピアノのための)\footnote{ 橋本晋哉による委嘱。}では、チューバがルイス・キャロル『不思議の国のアリス』の一節を語る。今回の「ゴシック・アンド・ロリータ」公演で演奏される、梅本による『~Me』(2022, ヴァイオリンのための)の連作において、不思議の国のアリスが引用されるのは、山根の出発点である『ケミカルロリイタ』の系譜を明確にするためでもある。その後の作品においても、山根は、『少女メランコリー』(2011, ヴァイオリンとトイピアノのための)、『ハラキリ乙女』(2012, 琵琶とオーケストラのための)などで、積極的に「少女」を描いてきた。

そんな山根の描く「少女」はどこか憂鬱だ。彼女の作品には、少女の自傷的な衝動が見て取れる。『少女メランコリー』では、「破壊的でグロテスクな思考回路」\footnote{\url{ https://akikoyamane.com/post/169459501812/girl-melancholy}}が描かれ、『ハラキリ乙女』では、「カッターなどの鈍い刃物」\footnote{\url{https://akikoyamane.com/post/168038351977/harakiri-maiden}}で、自ら肌を切り付ける。最初の委嘱作品の表記にみられるように、その少女は「ロリータ」ではなく、「ロリイタ」と、日本のロリータファッションの文脈であることが伺える。漂う死の匂いとロリイタ。そんな彼女の作品は、いかにもゴスロリ的なのである。

日本のゼロ年代思想や80年代以降のオタク論のように、言説の世界において、サブカルチャーとしての「ゴスロリ」は、「オタク」ほどには語られていない。山根は、今回のプロジェクトに関わるまで --- 自身がゴスロリ的な作品を作曲していたとはほとんど認識しておらず、自己を批評的に言語化してこなかった --- と梅本に語っているが、その内向的な性格は、まさに「ゴスロリの精神」のようである。オタク(の一部)が、現代美術や思想、言説をまとって、ある種の社交性を身に付けたのに比較して、ゴスロリは、メディア、言論への登壇を極力避けた。言葉を発しないということは、極めて強力な「防御」として、「美」を死守する手段なのかもしれない。

西洋音楽における日本式ポップアートを開拓した最重要人物である山根明季子が、長年の的外れな評論で、ほとんどまともな言語化をされることなくここまで来てしまったのには、上述のような一筋縄ではいかない複雑な要因があるのだろう。だがそれと同時に、私は絶対に、この価値ある革新の文脈を埋もれさせたくはない。なぜならば、私の、そして未来の芸術音楽の源流が間違いなくそこにあるからだ。そして、そのために、いま言語化が必要なのだ。我々 (mumyo) の活動の第一段階に「ゴシック・アンド・ロリータ」を置いたのは、まずそのエニグマを芸術でもって解き明かし、提示することにある。

ゴスロリを通して過去を顧みて、未来の世界を占う。情報の樹海を彷徨う現代の少女は、あらゆる装飾で防御し、その美でもって死を超越する。生まれもった肉体とあらゆる物質を自在に縫い合わせ、無限の可能性を持って変容すること。生まれ持ったものが物事の本質であるなどと、本質主義的に捉える時代は終わった。出自について語る意味と、出自の無意味さは共存する。書けば書くほど、聴けば聴くほどナンセンス。インベリッシュ・ミー! アリスの体は知でもって、その過剰な装飾によって変身する。 □ 

\newpage


\subsection*{小説「ロリータ」から43年後、 「ゴスロリ」は闇の姫服として誕生した}\label{suzuki}
{\small 文: 鈴木真理子}\footnote{雑誌『ケラ!』『ゴシック&ロリータバイブル』『ケラMANIAX』創刊編集長。現在フリーエディター&ライターとして、原宿系ファッション本のほか、美術館の公式本も手掛ける。
\url{https://twitter.com/marimari1961}
(Otona Alice Walk主宰。\url{https://twitter.com/otonaalice})}  

ロリータという言葉は1955年に発表されたV・ナボコフの小説「ロリータ」から生まれたもので、そもそも作品中で出てくるドロレスという12歳の少女の愛称だった。主人公の男性は中年だが、9~12歳の少女を偏愛する傾向があり、ドロレスに翻弄される様子が描かれた。この本がヒットし、さらに1961年S・キューブリックによって映画化もされ、西欧ではコケティッシュな魅力を持ち男性を手玉に取る少女達の性癖を「ロリータ・シンドローム」と呼び、その代表格はブリジット・バルドーとされていた。一方日本では幼女を含む少女を偏愛する男性の性癖を「ロリータ・コンプレックス」「ロリコン」と呼ぶようになった(和製英語)。

ファッション方面では日本では80年代のDCブランドブームを通し、現在ガーリーと呼ばれる服の他、ヴィヴィアン・ウエストウッドのコレクション(19世紀のクリノリンをミニ丈にしたミニクリニ等)他の発表などもあいまり、現在ロリータの範疇と考えられる服が複数ブランドで作られていった。

90年代後半になると、音楽シーンにマリリン・マンソンが登場、ファッションも含め世界中で一大ゴシックブームを興す。時同じくして日本ではヴィジュアル系バンドブームがピークを迎え、ゴシックテイストのバンドマンの姿が見られたが、特筆すべきはマリスミゼルのMana様で、自らの縦ロールヘア&黒いドレスの女装姿を「エレガント・ゴシック・ロリータ」と呼んだことだ。

「ゴスロリ」というワード自体は、ヴィジュアル系音楽もゴスロリ服もどちらも好きなユーザーの間で1998年頃自然発生的に生まれたものだが、それはバンドマンのコスプレ(服装の模倣)を指すものではない。あくまで着る人個人のファッションで、ゴシックテイストのあるロリータ服の着こなしを指すもので、ちなみに今もその定義は変わっていない。

姫袖やパニエ等、ゴスロリ装の中に詰め込まれたものは、ロココ&ヴィクトリア等近代西欧の富裕層の服装のアレンジで、肌の露出部分もごく少ない。ゴスロリ時代の到来によって、もはやここにロリコンと呼ばれる少女偏愛の性癖も、コケティッシュというワードも、さらに少女と呼ばれる実年齢とも全て関係は断ち切られてしまった。□


\newpage
%%本文はじめ
\thispagestyle{empty}


%\vspace{\stretch{1}}

\begin{center}
{\Large \ttfamily「ゴシック・アンド・ロリータ」} \\
{\ttfamily Gothic and Lolita} \\
\end{center}

%%\vspace{\stretch{1}}
\noindent {\ttfamily \Large mumyo}
\begin{description}
\item[出演] 成田達輝 (ヴァイオリニスト)
\item[作曲] 梅本佑利、山根明季子 
\end{description}

\noindent \\会場: BUoY (東京都足立区)  \\

\noindent 2023年4月2日 (日)  \\
第1回 開場 13:30 開演 14:00 \\
第2回 開場 16:30 開演 17:00  \\

%%\vspace{\stretch{1}}


\begin{table}[hbtp]
  \centering
  \begin{tabular}{lll}
  \hline
衣装協力 && 朝藤りむ / pays des fées   \\
イラスト & &  NABEchan  \\
    \hline
   助成  &   & 公益財団法人榎本文化財団    \\
            &  & 公益財団法人光山文化財団    \\ 
    \multicolumn{2}{r}{  \multirow{2}{*}{   \begin{minipage}{12mm }
      \centering
    \scalebox{0.18} {\includegraphics{ACT_logo/ACT_logo1.eps}} 
    \end{minipage} }  } & 公益財団法人東京都歴史文化財団 \\
%%             &                   & \hspace{2\zw} アーツカウンシル東京 \\ 
    &   & アーツカウンシル東京 [スタートアップ助成] \\
    \hline
  主催  &   &       合同会社無名 {\ttfamily (mumyo llc)}\\      
           &    & 〒102-0074 東京都千代田区九段南1-5-6 \\
  %%         &    & \hspace{6\zw} りそな九段ビル5階 \\
           &   &   \verb|https://mumyo.org/|  \\
  \end{tabular}
\end{table}


\vspace{-4mm}

\begin{center}
{\tiny \copyright 2023 mumyo llc.  \\   \vspace{-2mm}
 Typeset by H. Umemoto with Lua\LaTeX. Printed in Japan.
}
\end{center}

%%\vspace{\stretch{1}}

%%\hspace{2\zw}  






\end{document}
